\chapter{Materiais e Métodos} \label{ch:MM}

Este capítulo tem como finalidade descrever o projeto do sistema através da metodologia utilizada, de maneira a elucidar as várias facetas e componentes do sistema como um todo. O capítulo está dividido nas seguintes partes: Projeto de software e Tecnologias utilizadas.

\section{Projeto de Software} 

O projeto foi construído baseando-se em uma adaptação da metodologia Cascata em conjunto com a metodologia Iterativa e Incremental.

A metodologia Cascata foi cunhada por \citeonline{royce1970managing}. É composta por cinco etapas, onde a saída de cada etapa se torna a entrada da etapa posterior. Para isso, é necessário que a etapa anterior seja concluída e esteja correta para que se possa prosseguir com o desenvolvimento. As etapas do modelo são mostradas pela figura \ref{img:cascata}.

\imagem{0.6}{cascata}{Modelo de Desenvolvimento de \textit{Software} Cascata}{\citeonline{royce1970managing} (Traduzido)} 

A primeira etapa busca compreender os requisitos do sistema, que normalmente baseiam-se nas funcionalidades que precisam ser fornecidas, nas limitações e nos objetivos do software. Em seguida há o projeto do sistema, composto do modelo de dados, arquitetura do \textit{software}, interfaces de usuário e demais detalhes \cite{SITECASCATA1, SITECASCATA2}.

As etapas de implementação e testes dizem respeito à programação do \textit{software} em si e ao correto funcionamento das lógicas internas e funcionalidades externas do sistema, respectivamente. Por último têm-se a etapa de operação/manutenção, esta etapa resume-se a implantação, correção de erros não identificados durante a verificação, melhorias funcionais e suporte ao projeto \cite{SITECASCATA1, SITECASCATA2}.

A adaptação buscou simplificar o método cascata para adequá-lo à realidade do projeto. Pois devido à existência de um único projetista / desenvolvedor e ao tempo disponível para a execução do trabalho, algumas restrições do método não foram levadas em consideração, como por exemplo a rigidez sequencial na transição entre as etapas e o estabelecimento de requisitos explícitos no início do projeto.

Para solucionar as restrições citadas anteriormente utilizou-se o modelo Iterativo e Incremental. Este método prioriza o desenvolvimento/entrega de \textit{software} funcional em pequenos pedaços. A cada nova iteração o sistema recebe novas funcionalidades. Deste modo, o sistema é construído de forma incremental \cite{SITEINTERATIVOINCREMENTAL}.

Portanto, o sistema foi subdividido em diversas partes de acordo com os requisitos elencados, onde em cada subdivisão era escolhido um requisito para ser projetado, implementado, verificado e incrementado ao sistema.

\subsection{Requisitos do sistema} \label{subsec:requisitos}

Os requisitos de um sistema nada mais são do que um conjunto de descrições à respeito do que o sistema deve fazer, quais funcionalidades/serviços são oferecidos e quais são as restrições ou qual é o escopo do mesmo. Os requisitos de um sistema podem ser classificados como funcionais e não funcionais \cite{sommerville2011engenharia}.

Os requisitos ou características funcionais explicitam os serviços que o sistema deve oferecer, como ele deve reagir mediante determinadas entradas e qual seu comportamento em situações especificas \cite{sommerville2011engenharia, engsoftwilson}.

As características não funcionais são restrições aos serviços ou funcionalidades fornecidas pelo sistema. Ou seja, os requisitos não funcionais qualificam os requisitos funcionais. A qualificação pode ser referente ao \textit{software}, como a performance, usabilidade, confiabilidade, tolerância à falha, dentre outros. Ou pode ser referente ao processo de desenvolvimento, como tempos de entrega, método de desenvolvimento, e assim por diante \cite{introrequisitos}.

A seguir são listados os requisitos elencados para este projeto.


\subsubsection{Requisitos funcionais}

\begin{itemize}
    \item O sistema deve permitir o cadastro de novos usuários através da interface;
    
    \item O sistema deve permitir o \textit{login} dos usuários através da interface;

    \item O sistema deve permitir ao usuário visualizar as informações mais recentes sobre o clima;

    \item O sistema deve permitir ao usuário visualizar o histórico das variáveis climáticas mediante escolha do período a qual se refere;

	\item O histórico deve ser exibido na forma de gráficos, permitindo a seleção das variáveis mostradas;
	
	\item O sistema deve coletar os dados da estação meteorológica periodicamente e salvá-los na base de dados.
\end{itemize}

\subsubsection{Requisitos não funcionais}

\begin{itemize}
    \item O sistema deve ser de natureza multiplataforma, permitindo a consulta de dados tanto através da \textit{Web} quanto através de aplicativo \textit{Android};
    
    \item O sistema deve permitir a consulta de dados via \textit{internet} a qualquer momento, desde que haja conexão;

    \item Todas as consultas do sistema só podem ser efetuadas por usuários autenticados;

    \item O sistema deve possuir uma interface com boa experiência de usuário.

\end{itemize}

À partir dos requisitos elencados foram elaborados os casos de uso discutidos na subseção \ref{subsec:casosDeUso}.

\subsection{Arquitetura do sistema}

Inicialmente buscou-se compreender a arquitetura do sistema como um todo, levando em consideração o objetivo geral do projeto. Diante disso, identificou-se a necessidade de projetar uma arquitetura composta de um módulo responsável por obter os dados da estação meteorológica, um módulo com o papel de gerenciar os dados obtidos e por fim, um módulo para exibir as informações, ver figura \ref{img:modulos}.

\imagem{0.6}{modulos}{Módulos do sistema}{O Autor} 
\newpage
Para satisfazer a necessidade de operação da arquitetura da estação \textit{Vantage Vue™}, aderiu-se o \textit{socket TCP} como forma de captura dos dados pelo módulo responsável. Como o objetivo geral do projeto especifica que o sistema deve operar via \textit{internet}, o módulo de gerenciamento de dados foi projetado como uma API RESTful. O fato de ser adotada uma API para o sistema, resolve o problema de intercomunicação entre os módulos e a base de dados, pois atua como uma camada de abstração que gerencia toda a comunicação. Desta forma, o módulo de captura dos dados e o módulo de exibição podem se comunicar com o módulo de gerenciamento através de requisições HTTP, explanadas na subseção \ref{subsec:api}.

Além do mais, ambos os módulos de captura e de gerenciamento dos dados foram implementados utilizando a linguagem \textit{Javascript} e foram executadas no ambiente de execução \textit{Node.js}.
 
Ainda para satisfazer o objetivo geral do sistema, onde é previsto a visualização das informações por meio de uma aplicação \textit{Web} e também por meio de um aplicativo \textit{Android}, adotou-se o \textit{framework Ionic} para o desenvolvimento da interface de usuário.

A figura \ref{img:visaogeral}  demonstra o panorama geral do projeto e o contexto onde cada componente está inserido.

\imagem{0.165}{visaogeral}{Visão geral do sistema}{O autor}

%\newpage

A suíte de sensores da estação meteorológica (a) transmite as informações captadas por ondas de rádio para o seu console (b) que as exibe localmente em sua tela. O servidor de obtenção dos dados (d) requisita para o \textit{datalogger} as informações contidas no console. O \textit{datalogger} (c) captura essas informações e responde a requisição. O servidor de obtenção dos dados envia para o servidor de aplicação (e) as informações. A API então processa as informações colhidas, salva os dados no banco (f) e disponibiliza-os através da \textit{internet} (g) para a aplicação cliente (f).



\section{Tecnologias  utilizadas}

\subsection{\textit{Ionic Framework}} \label{subsec:Ionic}

O \textit{Ionic Framework} é um SDK que permite o desenvolvimento de aplicações multi-plataformas (\textit{Web, Android, iOS}) através de tecnologias \textit{Web} bem estabelecidas (HTML, CSS e JavaScript). O \textit{framework} é suportado pela tecnologia Angular \cite{SITEIONIC}.

Angular é uma plataforma, mantida pela Google, que torna fácil a criação de aplicações \textit{Web} cliente através de combinações entre \textit{templates} declarativos, injeção de dependências e ferramentas ponta-a-ponta \cite{SITEANGULAR}.

No caso do \textit{Ionic Framework}, o Angular é responsável pelo pilar base da arquitetura, a API de componentes. Os componentes são os blocos de construção da interface de usuário do aplicativo móvel \cite{SITEIONIC}.

A escolha desse \textit{framework} deve-se ao fato dele ser completamente gratuito, de código aberto (licenciado sob a licença MIT) e seguir o modelo de arquitetura \textit{cross-platform} híbrida (figura \ref{img:hybrid}), possibilitando a utilização de tecnologias \textit{Web} para construção de aplicativos e também a distribuição/instalação do \textit{APP} via o \textit{Market Place} da plataforma Android \cite{SITEIONIC}. O Ionic será empregado no desenvolvimento da interface de usuário do sistema, ou seja, o \textit{front-end} da aplicação.

\imagem{0.22}{hybrid}{Modelo de aplicação \textit{cross-platform} híbrida}{\citeonline{raj2012study} (Traduzido)}

\newpage

\subsection{Node.js} \label{subsec:NodeJs}

O \text{Node.js} é um ambiente de tempo de execução JavaScript assíncrono baseado em eventos, construído sobre o motor JavaScript V8 da Google, rodando do lado do servidor. Surgiu com o propósito de possibilitar a criação de aplicações de rede escaláveis. O paradigma baseado em eventos permite múltiplas conexões de diversos clientes ao servidor de forma eficiente, escalável e oferecendo muito mais controle para o desenvolvedor sobre as atividades de sua aplicação \cite{tilkov2010node}.
    
O \textit{Node.js} foi utilizado no escopo deste projeto para criação de um serviço responsável pela obtenção dos dados da estação meteorológica, empregando uma comunicação serial através de um \textit{socket TCP}, e pela inserção posterior desses no banco de dados. Há também outro serviço para a recuperação dos dados à serem usados na aplicação cliente, ou seja, o serviço terá o papel de fornecer uma API que por meio de requisições HTTP insira e recupere informações da base de dados.

A motivação da escolha dessa tecnologia é baseada nas duas características seguintes, o código do software é aberto e o fato da tecnologia também usar JavaScript torna natural a integração com aplicações \textit{cross-platform} que utilizam da mesma linguagem de programação. Além de facilitar a comunicação com a base de dados.


\subsection{MongoDB} \label{subsec:MongoDB}

O MongoDB é um banco de dados não relacional baseado em documentos, que armazena os dados em um formato semelhante ao Json. O software é grátis e de código aberto, licenciado sob a \textit{GNU Affero General Public License}. É adotado por grandes empresas como \textit{Adobe, Twitter, GitHub e Amazon}. A adoção desse banco de dados é justificada pela sua consistência, velocidade, escalabilidade e facilidade de utilização \cite{SITEMONGODB}. Além disso, os dados a serem trabalhados são do tipo chave-valor simples e não requerem uma base de dados relacional.

O banco de dados será usado majoritariamente para guardar as informações das variáveis climáticas provenientes das leituras da estação meteorológica, com a finalidade de manter um histórico das variações. Pois, o armazenamento disponível na estação é limitado e os dados são substituídos por novos à medida que as novas leituras ocorrem.



%\section{Cronograma} \label{sec:crono}

%A tabela \ref{tab:cronograma} mostra o cronograma de atividades a serem executadas para o TCC II, com base no calendário de 2017.2 da UNIVASF.

%\newpage
%\begin{table}[!thb]
%	%\huge
%    \centering
%    \caption{\label{tab:cronograma} Cronograma das atividades previstas para o TCC II}
%%    \begin{adjustbox}{max width=\textwidth}
%    \begin{tabular}{p{6.5cm}|c|c|c|c|c|c}
%    \toprule
%    \textbf{Atividade}                      & Nov & Dez & Jan & Fev & Mar & Abr \\ \hline
%    Implementar o banco de dados              & X    & X     &       &        &          &          \\ \hline
%    Desenvolver a API HTTP RESTful                      &   X   & X     &       &        &          &          \\ \hline
%    Implementar o serviço de captura de dados via \textit{socket} TCP        &      &      & X     &   X     &          &          \\ \hline
%    Desenvolver a aplicação \textit{Web/mobile} para exibição dos dados         &      &      & X     &   X     &     X     &          \\ \hline
 %   Teste do sistema            &      &       &       &        & X        &          %\\ \hline
 %   Escrita do TCC II                       &   X   & X     & X     & X      & X        & X        \\ \hline
%   Defesa do TCC II                        &      &       &       &        &          & X       \\
%    \bottomrule
 %   \end{tabular}
 %   \end{adjustbox}
%    \legend{\textbf{Fonte:} O autor.}
%\end{table}

%As primeiras etapas consistem no desenvolvimento das camadas de \textit{software} discutidas na seção \ref{sec:PS}, na ordem: camada de dados, camada da API, camada de obtenção dos dados e por fim a camada de usuário. 

%Na etapa de teste do sistema, serão efetuados testes para verificar a ocorrência de erros/\textit{bugs} e a fidelidade entre os dados mostrados pelo sistema e os dados mostrados pelo console da estação. Ademais, o \textit{software} cliente será verificado quanto a correta implementação dos casos de uso identificados e discutidos na subseção \ref{subsec:casosDeUso}.

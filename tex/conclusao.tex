\chapter{Conclusão e Trabalhos Futuros}

%Ao término das etapas apresentadas no cronograma da seção  espera-se obter um protótipo funcional de um sistema de monitoramento de variáveis meteorológicas cujos dados são provenientes de uma estação \textit{Vantage Vue™}, contendo todas as funcionalidades e características descritas no capítulo \ref{ch:MM}.

A preocupação em monitorar e prever as alterações climáticas do ambiente remete aos primórdios da humanidade. Com o passar do tempo e a consequente evolução da espécie, nos tornamos cada vez mais dependentes das condições do clima para garantir o sucesso no desenvolver de diversas atividades.

Neste trabalho foi elaborado e implementado um sistema de informação que coleta dados de uma estação meteorológica e a partir disso gera uma base de informações, permitindo consultas atuais e históricas das características medidas. As consultas são realizadas recorrendo à interface de usuário (\textit{Web / Android}) onde a \textit{internet} atua como meio de comunicação entre a base de dados e a aplicação cliente.

Compreende-se que os objetivos do trabalho foram atingidos de forma satisfatória. Pois, o protótipo dos sistema atendeu todos os requisitos propostos e mostrou resultados idênticos aos obtidos com o \textit{software} proprietário do fabricante da estação meteorológica em questão, no que diz respeito aos valores das variáveis climáticas.
 
\section{Trabalhos futuros}

Como trabalhos futuros pretende-se adicionar a funcionalidade ao sistema de monitorar os sensores da estação em tempo real, evitando necessidade do usuário agir e clicar em atualizar para verificar se há novos dados disponíveis.

Há também o objetivo de realizar a composição dos \textit{datasets} dos gráficos no servidor de aplicação (API), para então servi-los montados para a aplicação cliente,  melhorando assim a performance na exibição do histórico. 

Outro objetivo que se almeja é permitir monitorar diversas estações. A forma como o sistema foi implementado permite essa nova configuração com poucas linhas de código adicionais.
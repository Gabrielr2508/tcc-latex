\begin{capa}
\center
	\includegraphics[scale=0.6]{img/univasf.jpg}

	{\ABNTEXchapterfont\bfseries\large\imprimirinstituicao}

	\vspace*{5cm}
	{\ABNTEXchapterfont\bfseries\large\imprimirautor}

	\vfill
	{\ABNTEXchapterfont\bfseries\large\imprimirtitulo}
	\vfill
	\ABNTEXchapterfont\bfseries\large\imprimirlocal\\ \the\year

	\vspace*{1cm}

\end{capa}

\begin{folhaderosto}
\center
		{\ABNTEXchapterfont\bfseries\large\imprimirautor}
		\vspace*{\fill}

		{\ABNTEXchapterfont\bfseries\large\imprimirtitulo}
		\vspace*{\fill}

		{\hspace{.45\textwidth}
		\begin{minipage}{.5\textwidth}
			\SingleSpacing
			\imprimirpreambulo \\ \\

			{\imprimirorientadorRotulo~\imprimirorientador\par}
			{\imprimircoorientadorRotulo~\imprimircoorientador\par}

		\end{minipage}%
		\vspace*{\fill}}%
		\vspace*{\fill}
			\ABNTEXchapterfont\bfseries\large\imprimirlocal\\ \the\year
		\vspace*{1cm}
\end{folhaderosto}


%\begin{fichacatalografica}
%	\vspace*{\fill}					% Posição vertical
%	\hrule							% Linha horizontal
%	\begin{center}					% Minipage Centralizado
%	\begin{minipage}[c]{12.5cm}		% Largura
%
%	\imprimirautor
%
%	\hspace{0.5cm} \imprimirtitulo  / \imprimirautor. --
%	\imprimirlocal, \the\year-
%
%	\hspace{0.5cm} xx p. : il. (algumas color.) ; 30 cm.\\
%
%	\hspace{0.5cm} \imprimirorientadorRotulo~\imprimirorientador\\
%
%	\hspace{0.5cm}
%	\parbox[t]{\textwidth}{\imprimirtipotrabalho~--~\imprimirinstituicao,
%	\the\year.}\\
%
%	\hspace{0.5cm}
%		1. Palavra-chave1.
%		2. Palavra-chave2.
%		I. Orientador.
%		II. Universidade xxx.
%		III. Faculdade de xxx.
%		IV. Título\\
%
%	\hspace{8.75cm} CDU 02:141:005.7\\
%
%	\end{minipage}
%	\end{center}
%	\hrule
%\end{fichacatalografica}

\setlength{\ABNTEXsignwidth}{12cm}
\begin{folhadeaprovacao}
	\begin{center}
	    {\ABNTEXchapterfont\bfseries\large\imprimirinstituicao}
	    \vspace*{\fill}

	    {\ABNTEXchapterfont\bfseries\large FOLHA DE APROVAÇÃO}
	    \vspace*{\fill}

	    {\ABNTEXchapterfont\bfseries\large\imprimirautor}

	    \vspace*{\fill}\vspace*{\fill}
	    {\ABNTEXchapterfont\bfseries\large\imprimirtitulo}
	    \vspace*{\fill}

	    {\hspace{.45\textwidth}
		\begin{minipage}{.5\textwidth}
			\SingleSpacing
			\ABNTEXchapterfont\imprimirpreambulo \\ \\

			{\ABNTEXchapterfont\imprimirorientadorRotulo~\imprimirorientador\par}
			{\ABNTEXchapterfont\imprimircoorientadorRotulo~\imprimircoorientador\par}

		\end{minipage}%
	    \vspace*{\fill}}
	\end{center}

	\vspace*{\fill}	
	
	\begin{center}
			 \ABNTEXchapterfont\large Aprovado em: \_\_\_\_ de \_\_\_\_ de 2017
	\end{center}

	\vspace*{\fill}
	
	\begin{center}
			 \ABNTEXchapterfont\bfseries\large Banca Examinadora
	\end{center}
		
   \ABNTEXchapterfont\assinatura{Fábio Nelson de Sousa Pereira, Mestre, Universidade Federal do Vale do São Francisco}
	\ABNTEXchapterfont\assinatura{Jorge Luis Cavalcanti Ramos, Doutor, Universidade Federal do vale do São Francisco}
   \ABNTEXchapterfont\assinatura{Ricardo Argenton Ramos, Doutor, Universidade Federal do Vale do São Francisco}
	 \vspace*{\fill}

	 
\end{folhadeaprovacao}

% -- epígrafe
\newpage
\vspace*{\fill}
\begin{flushright}
		\textit{A minha família...}
\end{flushright}

\begin{agradecimentos}
	
	A minha mãe Josiane Gomes, meu pai Weliton de Carvalho e meus irmãos pelo esforço imensurável para que minha formação se concretizasse.
	
	Ao meu primo Manoel Rafael pelo apoio essencial durante a vida acadêmica.
	
	Aos meus amigos e companheiros de turma pela parceria, colaboração, compartilhamento de estresse e de fortes emoções durante o curso. Especialmente Esron Dtamar, Johnathan Alves, Gustavo Marques e Leonardo Cavalcante.
	
	Aos amigos conquistados no decorrer da caminhada, João Bastos, José Matias, Delmiro Daladier, Daniel Simião, Hallan Ferreira, Victor Silva, Antônio Noronha, Marlon Rocha e o pessoal restante do grupo do "AP" pela partilha da tarimba, casos e contos em mesas de bar.
	
	A Ana Letícia Menezes pelo companheirismo, paciência e pelo abraço caloroso.
	
	A meu tio Antônio Marcos pelo exemplo integridade e competência no aspecto pessoal e profissional.
	
	A meu primo Marco Antônio pela inspiração e minha tia Neiva Gomes por toda educação e amor dados à mim.
	
	Ao meu orientador, professor Fábio Nelson, pelos ensinamentos, dedicação e disposição.
	
	Aos professores Rômulo Câmara e Jairson Rodrigues pelas oportunidades de trabalhos extracurriculares que me enriqueceram muito como pessoa e profissional.
	
	A todos os demais professores que repassaram para mim um pouco de seu conhecimento e experiência, contribuindo de alguma forma para a realização deste trabalho e para minha formação.
	
	A minha madrinha Lúcia Ricarte, Joelma Duarte e Lurdinéia Guimarães por todo carinho e apoio de sempre.

	A Miguel Pérez Pasalodos por compartilhar seu conhecimento através de \textit{software open-source} e me aliviar de uma grande dor de cabeça.
	 
	Por fim, agradeço a todos aqueles que auxiliaram de alguma maneira a existência e o desenvolvimento de meu ser.

\end{agradecimentos}

% ---
% Epígrafe
% ---
\begin{epigrafe}
    \vspace*{\fill}
	\begin{flushright}
		Se pude enxergar a tão grande distância, foi subindo nos ombros de gigantes.\\
		 \vspace{\baselineskip}
		\textbf{Isaac Newton}\\
		\textbf{Carta à Robert Hooke, 1676}
	\end{flushright}
\end{epigrafe}
% ---


% ---
% RESUMOS
% ---
% resumo em português
\setlength{\absparsep}{18pt} % ajusta o espaçamento dos parágrafos do resumo
\begin{resumo}
Na região do Submédio do Vale do São Francisco, a organização social Biofábrica Moscamed Brasil, ou simplesmente Moscamed, sediada na cidade de Juazeiro-BA, é responsável pelo controle biológico da mosca-das-frutas e do mosquito-da-dengue. Esta é uma das maiores regiões produtoras de frutas do mundo. Entretanto, diversas culturas são atacadas pela praga da mosca-das-frutas ocasionando perdas consideráveis na produção. Por outro lado, a Moscamed atua também no combate ao mosquito \textit{Aedes Aegypti}, que é vetor de diversas doenças de grande importância quando se diz respeito à saúde pública. A Moscamed emprega um dos métodos mais eficazes no controle de ambos os insetos, a Técnica do Inseto Estéril (TIE). Porém, os ciclos de vida das duas espécies e consequentemente o êxito da técnica citada estão fortemente relacionados com as alterações climáticas do ambiente. Deste modo, visando auxiliar a tomada de decisões no que diz respeito aos processos relacionado à TIE e os demais processos da organização, o presente trabalho discorre sobre o desenvolvimento de um sistema de informação que fornece um mecanismo de obtenção e armazenamento e uma interface de visualização dos dados das variáveis climáticas provenientes de uma estação meteorológica \textit{Vantage Vue \textsuperscript{TM}}. O trabalho foi executado empregando-se tecnologias web, \textit{framework} para aplicativos multiplataforma e banco de dados não-relacional. O sistema resultante consiste em uma aplicação web / Android como a interface do usuário, uma aplicação que recupera informações da estação meteorológica e uma API para gerenciar os dados.

 \textbf{Palavras-chave}: \textit{Vantage Vue, Android, Ionic, Weather Link IP}, meteorologia.

\end{resumo}

% resumo em inglês
\begin{resumo}[Abstract]
\begin{otherlanguage*}{english}

In the sub-region of the São Francisco Valley, the social organization Biofábrica Moscamed Brasil, or simply Moscamed, based in the city of Juazeiro-BA, is responsible for the biological control of the medfly and the dengue mosquito. This is one of the largest fruit producing regions in the world. However, several crops are attacked by the medfly pests, causing considerable losses in production. On the other hand, Moscamed also acts in the fight against the mosquito \textit{Aedes Aegypti}, which is a vector of several diseases of great importance when it comes to public health. Moscamed employs one of the most effective methods in controlling both insects, the Sterile Insect Technique (SIT). However, the life cycles of both species and consequently the success of the cited technique are strongly related to the climate changes. In this way, in order to help decision-making regarding the TIE-related processes and the other processes of the organization, this work discusses the development of an information system that provides a retrieval and storage mechanism and a visualization interface of the climate variables data from a Vantage Vue\textsuperscript{TM} meteorological station. The work was performed using web technologies, framework for cross-platform applications and non-relational database. The resulting system consists of a web / Android application such as the user interface, an application that retrieves information from the weather station, and an API to manage the data.
	
	\vspace{\onelineskip}

	\noindent
	\textbf{Key-words}: \textit{Vantage Vue, Android, Ionic, Weather Link IP, meteorology}.

 \end{otherlanguage*}
\end{resumo}


% ---
% inserir lista de ilustrações
% ---
\begin{KeepFromToc}
\pdfbookmark[0]{\listfigurename}{lof}
\listoffigures
%\addcontentsline{toc}{chapter}{Lista de Figuras}
\cleardoublepage


% ---
% inserir lista de tabelas
% ---
\pdfbookmark[0]{\listtablename}{lot}
\listoftables
\cleardoublepage

% ---
%ajustar lista de códigos - alterar de figures para códigos
\makeatletter
\let\l@listing\l@figure
\def\newfloat@listoflisting@hook{\let\figurename\listingname}
\makeatother

% ---
% inserir lista de códigos
% @leolleocomp
% ---
\listoflistings

\end{KeepFromToc}
% inserir lista de abreviaturas e siglas
% ---
\begin{siglas}
	\item[API]      \textit{Application programming interface} - tradução: Interface de programação de aplicação
    \item[APP]		\textit{Application} - tradução: Aplicação
	\item[ART]      \textit{Android runtime} - tradução: Tempo de execução \textit{Android}
	\item[CECOMP]	Colegiado de engenharia de computação
	\item[CSS]      \textit{Cascading style sheet} - tradução: Tabela de estilos em cascata
    \item[GNU]		\textit{Gnu's not unix} - tradução: Gnu não é Unix
	\item[HAL]      \textit{Hardware abstration layer} - tradução: Camada de abstração de \textit{hardware} 
	\item[HTML]     \textit{Hypertext markup language} - tradução: Linguagem  de marcação de hipertexto
	\item[HTTP]     \textit{Hypertext transfer protocol} - tradução: Protocolo de transferência de hipertexto
	\item[IDE]      \textit{Integrated development environment} - tradução: Ambiente de desenvolvimento integrado
	\item[INMET]	Instituto nacional de meteorologia
    \item[JSON]	    \textit{Javascript object notation} - tradução: Notação de objeto Javascript
	\item[MIP]      Manejo integrado de pragas
    \item[MIT]		\textit{Massachusetts institute of technology} - tradução: Instituto de tecnologia de Massachusetts
	\item[REST]     \textit{Representational state transfer} - tradução: Transferência de Estado Representacional
	\item[SDK]		\textit{Software development kit} - tradução: Kit de desenvolvimento de programa
	\item[SGBD]		Sistema de gerenciamento de banco de dados
	\item[TCC]      Trabalho de conclusão de curso
    \item[TCP]		\textit{Transfer control protocol} - tradução: Protocolo de controle de transmissão 
	\item[TIE]		Técnica do inseto estéril
	\item[UR]	    Umidade relativa
	\item[URI]	    \textit{Uniform resource identifier} - tradução: Identificador de recurso univeforme
	\item[XML]      \textit{Extensible markup language} - tradução: Linguagem de marcação extensível
    
\end{siglas}

\pdfbookmark[0]{\contentsname}{toc} % inserir o sumario
\tableofcontents*
\cleardoublepage

% User commands
\usepackage{xcolor}
\newcommand\myworries[1]{\textcolor{red}{[#1]}}

% Pacotes fundamentais
\usepackage{lmodern}
%\usepackage{helvet}			% Usa a fonte Latin Modern
%\renewcommand{\familydefault}{\sfdefault} tira o serifado
\usepackage[T1]{fontenc}		% Selecao de codigos de fonte.
\usepackage[utf8]{inputenc}		% Codificacao do documento (conversão automática dos acentos)
\usepackage{indentfirst}		% Indenta o primeiro parágrafo de cada seção.
\usepackage{color}				% Controle das cores
\usepackage{tikz}				% Inclusão de gráficos
\usepackage{graphicx}			% Inclusão de gráficos
\usepackage{microtype} 			% para melhorias de justificação
% Pacotes adicionais, usados no anexo do modelo de folha de identificação
\usepackage{multicol}
\usepackage{multirow}
% Pacotes adicionais, usados apenas no âmbito do Modelo Canônico do abnteX2
\usepackage{lipsum}				% para geração de dummy text
% Pacotes de citações
\usepackage[brazilian,hyperpageref]{backref}	 % Paginas com as citações na bibl
\usepackage[alf,abnt-etal-list=3,abnt-etal-cite=3]{abntex2cite}	% Citações padrão ABNT
\usepackage{pdflscape}
\usepackage{listings}			% inserir codigo fonte
\usepackage{footnote}
\usepackage{pdfpages}
\usepackage{caption}

% [leolleo] meus pacotes
\usepackage{booktabs}
\usepackage{adjustbox}
\usepackage{subcaption}
\usepackage[labelfont=bf]{caption}
\usepackage{gensymb}
\usepackage{amsmath}
\usepackage{array}
\usepackage{float}
\usepackage{xcolor,colortbl}
\usepackage{longtable}
\usepackage{scalefnt}
% [leolleo] meus comandos (peguei do tex exchange)

\usepackage{tocloft}
% -- permite a adição de células especiais em tabelas
\newcommand{\specialcell}[2][c]{%
  \begin{tabular}[#1]{@{}c@{}}#2\end{tabular}}

\newcounter{equationset}
\newcommand{\equationset}[1]{% \equationset{<caption>}
  \refstepcounter{equationset}% Step counter
  \noindent\makebox[\linewidth]{Equação ~\theequationset: #1}
 }

\renewcommand{\ABNTEXchapterfont}{\fontseries{b}}
\renewcommand{\ABNTEXchapterfontsize}{\normalsize}

\renewcommand{\ABNTEXsectionfont}{\fontseries{m}}
\renewcommand{\ABNTEXsectionfontsize}{\normalsize}

\renewcommand{\ABNTEXsubsectionfont}{\fontseries{b}}
\renewcommand{\ABNTEXsubsectionfontsize}{\normalsize}

\renewcommand{\ABNTEXsubsubsectionfont}{\fontseries{m}}
\renewcommand{\ABNTEXsubsubsectionfontsize}{\normalsize}

%-------

% CONFIGURAÇÕES DE PACOTES
% Configurações do pacote backref
% Usado sem a opção hyperpageref de backref
\renewcommand{\backrefpagesname}{Citado na(s) página(s):~}
% Texto padrão antes do número das páginas
\renewcommand{\backref}{}
% Define os textos da citação
\renewcommand*{\backrefalt}[4]{
	\ifcase #1 %
		%Nenhuma citação no texto.%
	\or
		Citado na página #2.%
	\else
		Citado #1 vezes nas páginas #2.%
	\fi}%


% Configurações de aparência do PDF final
% alterando o aspecto da cor azul
\definecolor{blue}{RGB}{41,5,195}
% informações do PDF
\makeatletter
\hypersetup{
     	%pagebackref=true,
		pdftitle={\@title},
		pdfauthor={\@author},
    	pdfsubject={\imprimirpreambulo},
	    pdfcreator={LaTeX with abnTeX2},
		pdfkeywords={abnt}{latex}{abntex}{abntex2}{relatório técnico},
		colorlinks=true,			% false: boxed links; true: colored links
    	linkcolor=black,				% color of internal links
    	citecolor=black,				% color of links to bibliography
    	filecolor=black,			% color of file links
		urlcolor=black,
		bookmarksdepth=4
}
\makeatother
% ---

% ---
% Espaçamentos entre linhas e parágrafos
% ---

% O tamanho do parágrafo é dado por:
\setlength{\parindent}{1.3cm}

% Controle do espaçamento entre um parágrafo e outro:
\setlength{\parskip}{0.2cm}  % tente também \onelineskip

% ---
% compila o indice
% ---
\makeindex
% ---


\newcommand{\imagem}[4]
{%			\imagem{x.x}{nomeimg}{titulo}{fonte}
	\begin{figure}[!htb]
		\caption{\label{img:#2}#3}
		\begin{center}
			\includegraphics[scale=#1]{img/#2}
		\end{center}
        \legend{\textbf{Fonte:} #4}
	\end{figure}
}%

\newcommand{\xx} {$\bigotimes$}
\newcommand{\oo} {$\bigcirc$}

